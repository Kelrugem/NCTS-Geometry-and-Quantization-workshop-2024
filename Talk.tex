\documentclass[
aspectratio=3218, 
10pt
%, hyperref={pdfpagelabels=false}
]{beamer}

\usepackage{CJKutf8}
\usepackage[english]{babel}
%\usepackage{xcolor}
\usepackage{lmodern}
%\usepackage{amssymb}
%\usefonttheme{serif}
\usepackage[makeroom]{cancel} %for crossing symbols
\usepackage{leftindex} %For leftindex, making it possible to have nicely aligned left subscripts
%\usepackage[export]{adjustbox}
%\usepackage{calligra}
%\DeclareMathAlphabet{\mathcalligra}{T1}{calligra}{m}{n} %For small \mathcal letters
\makeatletter
\DeclareFontEncoding{LS1}{}{}
\DeclareFontSubstitution{LS1}{stix}{m}{n}
\DeclareMathAlphabet{\mathKel}{LS1}{stixscr}{m}{n}
\DeclareMathAlphabet{\mathcal}{LS1}{stixscr}{m}{n}
\usepackage{amsthm}
%\usepackage{amsmath}
%\usepackage{mathabx}
\usepackage{stmaryrd}
\usepackage{amsbsy}
\usepackage{dsfont}
\usepackage{mathtools} %für mathclap und coloneqq
%\usepackage{amsbsy}
\usepackage{mleftright} %Distanz zu \left \right weg
\usepackage{tikz-cd}

\usepackage{tabularx} %Automatic line break of tables using X instead c l r
%\usepackage{longtable} %table auf mehreren Seiten
%\usepackage{ltxtable} %Combination of both above
\usepackage{makecell} %For making larger cells in tables
\usepackage{colortbl}

%Für die ganzen Diagramme
\usepackage{pgfplots}

%Warning symbol
\usepackage{newunicodechar}

\newcommand\Warning{%  
\makebox[1.4em][c]{%  
\makebox[0pt][c]{\raisebox{0em}{\textbf{!}}}%  
\makebox[0pt][c]{\color{red}\LARGE$\bigtriangleup$}}}

\newunicodechar{⚠}{\Warning}

\usepackage{appendixnumberbeamer}

\usepackage{booktabs}
\usepackage[scale=2]{ccicons}

\usepgfplotslibrary{dateplot}
\usetikzlibrary{fit,
                shapes.arrows}

\usepackage{xspace}

%\usepackage{graphicx} %Für raisebox, vertical displacement of figures
\usetikzlibrary{decorations.markings, decorations.text,calc,arrows.meta}

\definecolor{Gray}{gray}{0.85}
%\usepackage[style=authortitle-icomp]{biblatex}
%\usepackage[babel,german=guillemets]{csquotes}

%\setcounter{tocdepth}{1}
%\setcounter{tocdepth}{5}
%\setcounter{secnumdepth}{4}
%\setcounter{secnumdepth}{5}
\usepackage[backend=biber, style=numeric]{biblatex}
\addbibresource{Literatur.bib}
\newcommand\fibtimes[2]{\mathbin{_{#1}\times_{#2}}}
%\newcommand{\footlineextra}[1]{
    %\begin{tikzpicture}[remember picture,overlay]
        %\node[yshift=2ex,anchor=south west] at (current page.south west) {\usebeamerfont{author in head/foot}\hspace{2ex}#1};
    %\end{tikzpicture}
%}

%\newcommand\insertreferences{}
%\setbeamertemplate{footline}{%
  %\leavevmode%
  %\hbox{%
  %\begin{beamercolorbox}[wd=.09\paperwidth, ht=5ex,dp=1ex,center, sep=1.4ex]{author in head/foot}%
    %\usebeamerfont{author in head/foot}
		%%\vfill
		%%Sources
		%%\vfill
		%Sources
  %\end{beamercolorbox}%
  %\begin{beamercolorbox}[wd=.91\paperwidth,ht=5ex,dp=1ex,center]{title in head/foot}%
    %\usebeamerfont{title in head/foot}
    %\insertreferences
%
  %\end{beamercolorbox}
%}
%}


%%% Transition slides
%\AtBeginSection[]{
  %\begin{frame}
  %%\vfill
	%\thispagestyle{empty}
  %\centering
  %\begin{beamercolorbox}[sep=8pt,center,shadow=true,rounded=true]{title}
    %\usebeamerfont{title}\insertsection\par%
  %\end{beamercolorbox}
  %%\vfill
  %\end{frame}
%}

\title{Curved Yang-Mills-Higgs theories}   
\subtitle{}   
\author{Simon-Raphael Fischer, \textit{based on joint works with Camille Laurent-Gengoux, and with Mehran Jalali Farahani, Hyungrok Kim (\begin{CJK*}{UTF8}{bkai}金炯錄\end{CJK*}), Christian Sämann}} 
\institute{
\begin{figure}
	\centering
		\includegraphics[width=.50\textwidth]{Logo_Uni_Göttingen_2022.png}
	\label{fig:NCTS}
\end{figure}
\begin{center}
%Mathematical Institute
\end{center}
}
\date{} 
%\date{Le lundi 31 mai 2021} 

% zusaetzlich ist das usepackage{beamerthemeshadow} eingebunden 
%\usepackage{beamerthemeIlmenau}
%\usepackage{beamerthemeshadow}
%\usepackage{beamerthemeDarmstadt}
%\usetheme{Arguelles}
\usetheme{metropolis}
%\usepackage[orientation=landscape,size=custom,width=16,height=9,scale=0.5,debug]{beamerposter}

%  \beamersetuncovermixins{\opaqueness<1>{25}}{\opaqueness<2->{15}}
%  sorgt dafuer das die Elemente die erst noch (zukuenftig) kommen 
%  nur schwach angedeutet erscheinen 
\beamersetuncovermixins{\opaqueness<1>{25}}{\opaqueness<2->{15}}
% klappt auch bei Tabellen, wenn teTeX verwendent wird\ldots

\beamertemplatenavigationsymbolsempty %Damit sind die kleinen Navigationssymbole unten weg

%\usesectionheadtemplate{}{}
%\usesubsectionheadtemplate{}{}

\def\be{\begin{equation}}
\def\ee{\end{equation}}
\def\bs{\begin{subequations}}
\def\es{\end{subequations}}
\def\ba#1\ea{\begin{align}#1\end{align}}
\def\bes{\begin{equation*}}
\def\ees{\end{equation*}}
\def\bas#1\eas{\begin{align*}#1\end{align*}}

\AtBeginEnvironment{remark}{%
  \setbeamercolor{block title}{use=example text,fg=black,bg=yellow!75!black}
  \setbeamercolor{block body}{parent=normal text,use=block title example,bg=yellow!10}
}
\AtBeginEnvironment{motivation}{%
  \setbeamercolor{block title}{use=example text,fg=black,bg=yellow!75!black}
  \setbeamercolor{block body}{parent=normal text,use=block title example,bg=yellow!10}
}
%\AtBeginEnvironment{example}{%
  %\setbeamercolor{block title}{use=example text,fg=black,bg=green!75!black}
  %\setbeamercolor{block body}{parent=normal text,use=block title example,bg=green!10}
%}

\renewcommand{\qedsymbol}{}
\theoremstyle{plain}
\newtheorem{conjecture}[theorem]{Conjecture}
\newtheorem{proposition}[theorem]{Proposition}
%\newtheorem{definition}[theorem]{Definition}
\theoremstyle{remark}
\newtheorem*{remark}{Remarks}
\newtheorem*{gedankenexperiment}{Gedankenexperiment}
\newtheorem*{idea}{Idea}
\newtheorem*{motivation}{Motivation}
\newtheorem*{summary}{Summary}
\newtheorem*{situation}{Situation}
\newtheorem*{lab}{Situation: Lie algebra bundles}
\newtheorem*{question}{Question}
\newtheorem*{fieldredefinition}{Field Redefinition}
\newtheorem*{construction}{Construction}
\newtheorem*{aim}{Aim}
\newtheorem*{BackToTheRoots}{Back to the roots}

\AtBeginEnvironment{BackToTheRoots}{%
  \setbeamercolor{block title}{use=example text,fg=black,bg=pink!75!black}
  \setbeamercolor{block body}{parent=normal text,use=block title example,bg=pink!10}
}
\AtBeginEnvironment{gedankenexperiment}{%
  \setbeamercolor{block title}{use=example text,fg=black,bg=pink!75!black}
  \setbeamercolor{block body}{parent=normal text,use=block title example,bg=pink!10}
}
\AtBeginEnvironment{idea}{%
  \setbeamercolor{block title}{use=example text,fg=black,bg=pink!75!black}
  \setbeamercolor{block body}{parent=normal text,use=block title example,bg=pink!10}
}

%\theoremstyle{definition}
%\newtheorem{definition}[theorem]{Definition}
%\newtheorem*{SecondIn}{Second Inequality}

%mathrm mit mathup ersetzen, damit die font passt
\renewcommand\familydefault{\sfdefault} %comment to see the difference
\DeclareMathAlphabet      {\mathup}{OT1}{\familydefault}{m}{n}


\begin{document}


\begin{frame}
\thispagestyle{empty}
\titlepage
\end{frame} 


{
\setbeamertemplate{footline}{}
\begin{frame}
%\frametitle{Table of contents}
%\tableofcontents
\thispagestyle{empty}
\begin{figure}[htbp]
	\centering
		\includegraphics[width=1\textwidth]{Research circles I.pdf}
\end{figure}
\end{frame} 
}

\section{Curved Yang-Mills-Higgs theory}

\subsection{Motivation}



\begin{frame}{Motivation: Covariantisation of Yang-Mills(-Higgs) theory}
\begin{tikzpicture}
\coordinate (a) at (0, 0);
\coordinate (b) at (15, 0);
\path[->, line width=1mm] (a) edge node[above] {Covariantization} (b);
\end{tikzpicture}
{\renewcommand{\arraystretch}{2}
\begin{table}[h!]
		\begin{tabularx}{\textwidth}{c c c}
			\rowcolor{gray}
			Classical theory & Covariantised flat theory & Curved Theory \\
			Vector space $V$ & Trivial vector bundle $M \times V$ & Vector bundle $V \to M$ \\
			\rowcolor{Gray}
			$\frac{\partial}{\partial x^i}$ & Canonical flat connection $\nabla^0$ & Vector bundle connection $\nabla$ \\ 
			\multicolumn{1}{X}{Coordinate changes may lead to extra terms} & 
			\multicolumn{2}{c}{Coordinate expressions form-invariant under coordinate changes}
		\end{tabularx}
\end{table}}
\vspace{-30pt}
\begin{tikzcd}[ampersand replacement=\&]
\phantom{Coordinate changes may lead to extra terms} \arrow[bend right, equal]{r} \& \phantom{Coordinate}
	\end{tikzcd}
 \end{frame}

\setbeamertemplate{frame footer}{S.-R.\ Fischer. \textit{Integrating curved Yang–Mills gauge theories}, arXiv: 2210.02924, 2022. \newline 
S.-R.\ Fischer. \textit{Geometry of curved Yang–Mills–Higgs gauge theories}, Ph.D.\ thesis, Institut Camille Jordan [Villeurbanne], France, U. Geneva, Switzerland, 2021; doi: 10.13097/archive-ouverte/unige:152555}

\begin{frame}{Curved Yang-Mills gauge theory (curved YM theory)}
\begin{tikzpicture}
\coordinate (a) at (0, 0);
\coordinate (b) at (15, 0);
\path[->, line width=1mm] (a) edge node[above] {Covariantization} (b);
\end{tikzpicture}
{\renewcommand{\arraystretch}{2}
\begin{table}[h!]
		\begin{tabularx}{\textwidth}{c c c}
			\rowcolor{gray}
			YM theory & Covariantised YM theory & Curved YM theory \\
			Lie group $G$ & Trivial Lie group bundle $M \times G$ & Lie group bundle $G \to M$ \\
			\rowcolor{Gray}
			Maurer-Cartan form & Fibre-wise Maurer-Cartan & Multiplicative YM connection \\ 
			\multicolumn{1}{X}{Field redefinitions lead to extra terms in gauge transformations and field strength} & 
			\multicolumn{2}{c}{\makecell{Expressions form-invariant under field redefinitions, \\ \textbf{but curvature transforms non-trivially}}}
		\end{tabularx}
\end{table}}
\vspace{-30pt}
\begin{tikzcd}[ampersand replacement=\&]
\phantom{Coordinate changes may lead to extra terms} \arrow[bend right, equal]{r} \& \phantom{Coordinate}
	\end{tikzcd}
\end{frame}

\setbeamertemplate{frame footer}{S.-R.\ Fischer, M.\ Jalali Farahani, H.\ Kim, and C.\ Saemann. \textit{Adjusted connections I: Differential cocycles for principal groupoid bundles with connection}, arXiv: 2406.16755, 202. \newline 
S.-R.\ Fischer. \textit{Geometry of curved Yang–Mills–Higgs gauge theories}, Ph.D.\ thesis, Institut Camille Jordan [Villeurbanne], France, U. Geneva, Switzerland, 2021; doi: 10.13097/archive-ouverte/unige:152555}

\begin{frame}{Curved Yang-Mills-Higgs theory (curved YMH theory)}
\begin{tikzpicture}
\coordinate (a) at (0, 0);
\coordinate (b) at (15, 0);
\path[->, line width=1mm] (a) edge node[above] {Covariantization} (b);
\end{tikzpicture}
{\renewcommand{\arraystretch}{2}
\begin{table}[h!]
		\begin{tabularx}{\textwidth}{c c c}
			\rowcolor{gray}
			YMH theory & Covariantised YMH theory & Curved YMH theory \\
			Lie group $G$ with right-action on $N$ & Action groupoid $N \times G$ & Lie groupoid $G \rightrightarrows N$ \\
			\rowcolor{Gray}
			Maurer-Cartan form & Fibre-wise Maurer-Cartan & Covariant adjustments \\ 
			\multicolumn{1}{X}{Field redefinitions lead to extra terms in gauge transformations and field strength} & 
			\multicolumn{2}{c}{\makecell{Expressions form-invariant under field redefinitions, \\ \textbf{but curvature transforms non-trivially}}}
		\end{tabularx}
\end{table}}
\vspace{-30pt}
\begin{tikzcd}[ampersand replacement=\&]
\phantom{Coordinate changes may lead to extra terms} \arrow[bend right, equal]{r} \& \phantom{Expressions form-invariant}
	\end{tikzcd}
\end{frame}



{
\setbeamertemplate{frame footer}{Ana Cannas Da Silva and Alan Weinstein. \textit{Geometric models for noncommutative algebras}, volume 10. American Mathematical Soc., 1999.}

\begin{frame}
\begin{center}
	\begin{tikzcd}[ampersand replacement=\&]
\mathcal{G} \arrow[d, shift left, "s"]\arrow["t", swap, shift right]{d} \\
	N
	\end{tikzcd}
\end{center}

\begin{definition}[Lie groupoids]
\vspace{.5pt}
$\mathcal{G}$ a \textbf{Lie groupoid} if there are surjective submersions $s, t \colon \mathcal{G} \to N$,  \textit{source} and \textit{target}, respectively, and a smooth \textit{multiplication map} $\mathcal{G} \fibtimes{s}{t} \mathcal{G} \to \mathcal{G}$ such that 
\bes
s(g'g) = s(g), \qquad t(g'g) = t(g')
\ees
for all $(g', g) \in \mathcal{G} \fibtimes{s}{t} \mathcal{G}$ (i.e.\ $s(g') = t(g)$), satisfying the typical expected properties, that is,
\bas
\text{Associativity:} && (g'' g') g &= g'' (g'g),
\\
\text{Units:} && g e_{s(g)} = g, \quad&\quad e_{t(g)} g = g,
\\
\text{Inverse:} && g^{-1} g = e_{s(g)}, \quad&\quad g g^{-1} = e_{t(g)}
\eas
for all $(g'', g', g) \in \mathcal{G} \fibtimes{s}{t} \mathcal{G} \fibtimes{s}{t} \mathcal{G}$, where one requires the existence of the \textit{unit} $e$ as a global section of both, $s$ and $t$, and the \textit{inverse} $g^{-1} \in \mathcal{G}$ of $g$.
\end{definition}
\end{frame}

\begin{frame}
\begin{center}
	\begin{tikzcd}[ampersand replacement=\&]
x'' \& \arrow[l, swap, "g'", bend right] x' \& \arrow[l, swap, "g", bend right] \arrow[ll, swap, "g'g", bend right=70] x
	\end{tikzcd}
\end{center}
\end{frame}

\begin{frame}
\begin{center}
	\begin{tikzcd}[ampersand replacement=\&]
* \& \arrow[l, swap, "g'", bend right] * \& \arrow[l, swap, "g", bend right] \arrow[ll, swap, "g'g", bend right=70] *
	\end{tikzcd}
\end{center}
%\pause
\begin{example}[Lie groups]
\vspace{.5pt}
Lie groups $G$
\begin{center}
	\begin{tikzcd}[ampersand replacement=\&]
G \arrow[d, shift left, "s"]\arrow["t", swap, shift right]{d} \\
	\{*\}
	\end{tikzcd}
\end{center}
\end{example}
\end{frame}

\begin{frame}
\begin{center}
	\begin{tikzcd}[ampersand replacement=\&]
x \& \arrow[l, swap, "g'", bend right] x \& \arrow[l, swap, "g", bend right] \arrow[ll, swap, "g'g", bend right=70] x
	\end{tikzcd}
\end{center}

\begin{example}[Lie group bundles (LGBs)]
\vspace{.5pt}
LGB $\pi_G \colon G \to M$
\begin{center}
	\begin{tikzcd}[ampersand replacement=\&]
G \arrow[d, shift left, "\pi_G"]\arrow["\pi_G", swap, shift right]{d} \\
	M
	\end{tikzcd}
\end{center}
\end{example}
\end{frame}


\begin{frame}
\begin{center}
	\begin{tikzcd}[ampersand replacement=\&]
x \& \arrow[l, swap, "g'" pos=.42, bend right] x \cdot q \& \arrow[l, swap, "g" pos= .45, bend right] \arrow[ll, swap, "g'g", bend right=70] x \cdot q q'
	\end{tikzcd}
\end{center}


\begin{example}[Action groupoid (trivial)]
\vspace{.5pt}
Lie group $G$ with action $\Psi \colon N \times G \to N$, $(p, q) \mapsto p \cdot q$, on $N$. 
\begin{center}
	\begin{tikzcd}[ampersand replacement=\&]
N \times G \arrow[d, shift left, "\Psi"]\arrow["\mathup{pr}_1", swap, shift right]{d} \\
	N
	\end{tikzcd}
\bas
(x, q) ~ (x \cdot q, q')&=(x, q q')~,
\\
e_x &= \left(x, e \right)~,
\\
(x, q)^{-1} &= \left(x \cdot q, q^{-1} \right)
\eas
\end{center}
\end{example}
\end{frame}

\setbeamertemplate{frame footer}{Kirill C.~H.\ Mackenzie. \textit{General theory of {L}ie groupoids and {L}ie algebroids}, London Mathematical Society Lecture Note Series, 213:xxxviii+501, 2005.}

\begin{frame}
\begin{center}
	\begin{tikzcd}[ampersand replacement=\&]
P \arrow["\Phi", swap]{rd} \& \arrow[bend right]{l} \mathcal{G} \arrow[d, shift left, "s"]\arrow["t", swap, shift right]{d} \\
	 \& N
	\end{tikzcd}
\end{center}

\begin{definition}[Groupoid right-action]\vspace{.5pt}
A \textbf{right-action} is a smooth map $P \fibtimes{\Phi}t \mathcal{G} \to P$ such that
\bas
\Phi(p \cdot g') &= s(g'),
\\
(p \cdot g') \cdot g &= p \cdot (g' g),
\\
p \cdot e_{\Phi(p)} &= p
\eas
for all $(p, g', g) \in P \fibtimes{\Phi}t \mathcal{G} \fibtimes ts \mathcal{G}$.
\end{definition}
\end{frame}

\setbeamertemplate{frame footer}{I.~Moerdijk and J.~Mr{\v c}un. \textit{Introduction to foliations and Lie groupoids}, Cambridge University Press, 2003.}

\begin{frame}
\begin{center}
	\begin{tikzcd}[ampersand replacement=\&]
P \arrow["\Phi", swap]{rd} \arrow["\pi", swap]{d} \& \arrow[bend right]{l} \mathcal{G} \arrow[d, shift left, "s"]\arrow["t", swap, shift right]{d} \\
	M \& N
	\end{tikzcd}
\end{center}

\begin{definition}[Principal groupoid-bundles]\vspace{.5pt}
$\pi\colon P \to M$ surjective submersion is a \textbf{principal $\mathcal{G}$-bundle} if
\bes
\pi(p \cdot g) = \pi(p)
\ees
for all $(p, g) \in P \fibtimes{\Phi}t \mathcal{G}$,
and if
\bas
P \fibtimes{\Phi}t \mathcal{G} &\to P \fibtimes\pi\pi P,
\\
(p, g) &\mapsto (p, p\cdot g)
\eas
is a diffeomorphism.
\end{definition}
\end{frame}

\setbeamertemplate{frame footer}{}

\begin{frame}{Ehresmann connection}
\begin{center}
	\begin{tikzcd}[ampersand replacement=\&]
P \arrow["\Phi", swap]{rd} \arrow["\pi", swap]{d} \& \arrow[bend right]{l} \mathcal{G} \arrow[d, shift left, "s"]\arrow["t", swap, shift right]{d} \\
	M \& N
	\end{tikzcd}
\end{center}

\begin{remark}
Infinitesimal action:
\bas
\mathup{T}P \fibtimes{\mathup{D}\Phi}{\mathup{D}t} \mathup{T}\mathcal{G}
&\to 
\mathup{T}P,
\\
(X, Y) &\mapsto X \cdot Y.
\eas
\pause
For $r_g(p) \coloneqq p \cdot g$ with $\Phi(p) = t(g)$, its infinitesimal version corresponds to
\bes
\mathup{D}r_g(X) = X \cdot 0
\ees
for all $X$ with $\mathup{D}\Phi(X) = 0$.
\end{remark}
\end{frame}

\setbeamertemplate{frame footer}{I.~Moerdijk and J.~Mr{\v c}un. \textit{Introduction to foliations and Lie groupoids}, Cambridge University Press, 2003. \newline
D.~Signori and M.~Sti\'{e}non. \textit{On nonlinear gauge theories}, J.\ Geom.\ Phys.\ {\bf 59} 1063, 2009.}

\begin{frame}{Ehresmann connection}
\begin{center}
	\begin{tikzcd}[ampersand replacement=\&]
P \arrow["\Phi", swap]{rd} \arrow["\pi", swap]{d} \& \arrow[bend right]{l} \mathcal{G} \arrow[d, shift left, "s"]\arrow["t", swap, shift right]{d} \\
	M \& N
	\end{tikzcd}
\end{center}

Infinitesimal $\mathcal{G}$-action on $P$:
\bas
\mathup{T}P \fibtimes{\mathup{D}\Phi}{\mathup{D}t} \mathup{T}\mathcal{G}
&\to 
\mathup{T}P,
\\
(X, Y) &\mapsto X \cdot Y.
\eas

\begin{idea}
Horizontal distribution $\mathup{H}P$ (w.r.t.\ $\pi$) with
\bes
\mathup{D}\Phi ( \mathup{H}P ) = 0.
\ees
\end{idea}

\end{frame}

\setbeamertemplate{frame footer}{S.-R.\ Fischer. \textit{Integrating curved Yang–Mills gauge theories}, arXiv: 2210.02924, 2022. \newline 
S.-R.\ Fischer. \textit{Geometry of curved Yang–Mills–Higgs gauge theories}, Ph.D.\ thesis, Institut Camille Jordan [Villeurbanne], France, U. Geneva, Switzerland, 2021; doi: 10.13097/archive-ouverte/unige:152555}


\begin{frame}{We lose covariantization}


%\begin{center}
{\renewcommand{\arraystretch}{2}
\begin{table}[h!]
\centering
		\begin{tabularx}{\textwidth}{c c c}
			%\rowcolor{gray}
			 & \cellcolor{gray} YM theory & \cellcolor{gray} Covariantised YM theory\\
			\phantom{assssssssdasdasdasd} & Lie group $G$ & Trivial Lie group bundle $M \times G$ \\
			%\rowcolor{Gray}
			 & \cellcolor{Gray} Maurer-Cartan form & \cellcolor{Gray} Fibre-wise Maurer-Cartan
		\end{tabularx}
\end{table}}

\begin{center}
\vspace{-25pt}
\begin{tikzcd}[ampersand replacement=\&]
\phantom{Maurer-Cartan form} \arrow[bend right, equal]{r} \& \phantom{Fibre-wise Maurer-Cartan}
	\end{tikzcd}
\end{center}
%\end{center}
	
	
\begin{center}
\Warning We would lose the equivalence to the covariantised theory where we set $\Phi = \pi$ \Warning
\end{center}

\begin{center}
	\begin{tikzcd}[ampersand replacement=\&]
P \arrow["\Phi"]{rd} \arrow["\pi", swap]{d} \& \arrow[bend right]{l} M \times G \arrow{d} \\
	M \arrow[equal]{r} \& M
	\end{tikzcd}
\end{center}
\end{frame}

\setbeamertemplate{frame footer}{
S.-R.\ Fischer, M.\ Jalali Farahani, H.\ Kim, and C.\ Saemann. \textit{Adjusted connections I: Differential cocycles for principal groupoid bundles with connection}, arXiv: 2406.16755, 202. \newline 
S.-R.\ Fischer. \textit{Integrating curved Yang–Mills gauge theories}, arXiv: 2210.02924, 2022.}

\begin{frame}{New idea!}
\begin{center}
	\begin{tikzcd}[ampersand replacement=\&]
P \arrow["\Phi", swap]{rd} \arrow["\pi", swap]{d} \& \arrow[bend right]{l} \mathcal{G} \arrow[d, shift left, "s"]\arrow["t", swap, shift right]{d} \\
	M \& N
	\end{tikzcd}
\end{center}

\begin{idea}[Ehresmann connection on $P$]
Equip $\mathcal{G}$ with a horizontal distribution $\mathup{H}\mathcal{G}$ (w.r.t.\ $t$). An Ehresmann connection on $P$ is a horizontal distribution $\mathup{H}P$ (w.r.t.\ $\pi$) so that the infinitesimal $\mathcal{G}$-action on $P$ restricts 
\bas
\mathup{H}P \fibtimes{\mathup{D}\Phi}{\mathup{D}t} \mathup{H}\mathcal{G}
&\to 
\mathup{H}P.
\eas
\pause
Invariance then via the \textbf{modified right-pushforward $\mathcal{r}_{g*}$}
\bes
\mathcal{r}_{g*}(X) \coloneqq X \cdot Y
\ees
for all $X \in \mathup{T}_pP$, where $Y \in \mathup{H}_g\mathcal{G}$ is the unique lift of $\mathup{D}_p\Phi(X)$.
\end{idea}

\end{frame}


\section{Singular Foliations}

%{
%\section[Singular Foliations]{Applications: Classifying singular foliations \\(joint work w/ Camille Laurent-Gengoux)}
%}
\subsection{Why foliations?}
{
%\setbeamertemplate{footline}
{

}
\begin{frame}{Example of a singular foliation}
\begin{center}
\begin{tikzpicture}
\coordinate (O) at (0,0);
%\foreach \j in {1,...,3} \draw (O) circle (3.5-\j);
%\foreach \k/\text in {0/Should be here any!,1/There is a way?,2/Wee} \draw[decoration={text along path,reverse path,text align={align=center},text={\text}},decorate] (2.6-\k,0) arc (0:180:2.6-\k);
\foreach \k in {1,...,4}\pgfmathparse{12*\k} \draw[fill=blue!\pgfmathresult] (O) circle (3.6-0.8*\k) node at (0, 3) {$\mathbb{S}^n$};
%\foreach \k/\text in {0/{$S^n$},1/,2/,3/} \draw[decoration={text along path,reverse path,text align={align=center},text={\text}},decorate] (2.9-0.8*\k,0) arc (0:180:2.9-0.8*\k);
\fill (O) circle[radius=2pt];
%\begin{scope}[xshift=2.2cm, yshift=-1.8cm]
%\begin{axis}[ scale = 1,
            %hide axis,
            %%axis lines=middle,
%%            axis on top,
%%            axis line style={blue,dashed,thick},
%%            ymin=-2,ymax=2,
%%            xmin=-2,xmax=2,
%%            zmin=-2,zmax=2,
            %samples=40,
            %domain=0:360,
            %y domain=0:1.25,clip=false
        %]
        %\addplot3 [surf, shader=flat, draw=black, fill=gray!10!white, z buffer=sort]
           %({sin(x)*y}, {cos(x)*y}, {(y^2-1)^2});
        %\draw[blue,thick,dashed] (axis cs:0,0,0) -- (axis cs:1,0,0)
                    %node[below,font=\footnotesize]{};
        %\draw[blue,thick,-stealth] (axis cs:1,0,0) -- (axis cs:1.3,0,0)
                    %node[above,font=\footnotesize]{};
        %\draw[blue,thick,dashed] (axis cs:0,0,0) -- (axis cs:0,-1,0)
                    %node[left=2mm,font=\footnotesize]{}; %{Label} am Ende 
        %\draw[blue,thick,-stealth] (axis cs:0,-1,0) -- (axis cs:0,-1.5,0)
                    %node[right=1mm,font=\footnotesize]{};
        %\draw[blue,thick,dashed] (axis cs:0,0,0) -- (axis cs:0,0,1)
                    %%node[left=2mm,font=\footnotesize]{$\phi_{\text{RE}}$}
                    %;
        %\draw[blue,thick,-stealth] (axis cs:0,0,1) -- (axis cs:0,0,1.3);
%\end{axis}
%\end{scope}
%\draw[line width=2mm,>={Triangle[length=3mm,width=5mm]},->] (2.6,0) -- (3.8,0);
\end{tikzpicture}
\end{center}
\end{frame}
}

{
%\setbeamertemplate{footline}{}
\begin{frame}{Foliations are widespread}
\textbf{Singular Foliations:}

\begin{itemize}
	\item Gauge Theory
	\item Poisson Geometry \newline (Singular foliation of symplectic leaves)
	\item Lie groupoids and algebroids
	\item Dirac structures
	\item Generalised complex manifolds
	\item Non-commutative geometry
	\item $\dotsc$
\end{itemize}

\end{frame}
}




\subsection{Idea: Relation to gauge theory}

\setbeamertemplate{frame footer}{{Camille Laurent-Gengoux and Leonid Ryvkin, The holonomy of a singular leaf, \textit{Selecta Mathematica 28}, no.\ 2, 45, 2022.}}

\begin{frame}{Unique transverse structure}
\begin{figure}[htbp]
	\centering
		\includegraphics[width=.8\textwidth]{Foliation connection.png}
	%\caption{$\mathcal{F}$-connections}
	\label{fig:Foliation connection}
\end{figure}

\end{frame}

\setbeamertemplate{frame footer}{{Source of the existence of connection on normal bundle: Camille Laurent-Gengoux and Leonid Ryvkin, The holonomy of a singular leaf, \newline \textit{Selecta Mathematica 28}, no.\ 2, 45, 2022.}}

\begin{frame}{How to classify singular foliations?}
\begin{figure}[htbp]
	\centering
		\includegraphics[width=0.4\textwidth]{Foliation connection.png}
	%\caption{$\mathcal{F}$-connections}
	\label{fig:Foliation connection Zwei}
\end{figure}

\begin{remark}[{[C.\ L.-G., S.-R.\ F.]}]
There is a connection on the normal bundle of a leaf $L$ preserving the foliation!
\begin{itemize}
	\item Transverse structure is unique: Classify singular foliation $\mathcal{F}$ like a bundle!
	\item Connection is a multiplicative Yang-Mills connection: Use curved gauge theory!
\end{itemize}
\end{remark}

\end{frame}

\subsection{Result}
\setbeamertemplate{frame footer}{}
{
%\setbeamertemplate{footline}{}
%\begin{frame}{Result}
%\begin{block}{Theorem ([C.\ L.-G., S.-R.\ F.])}
%\vspace{.5pt}
%Formal singular foliations with leaf $L$ and transverse model $(\mathbb{R}^d, \tau_l)$ are equivalent to:
%\begin{itemize}
	%\item A Galois cover $L'$ over $L$ with structural group $K$
	%\item A short exact sequence of groups
	%\begin{center}
		%\begin{tikzcd}[ampersand replacement=\&]
		%\mathup{Inner}(\tau_l) \arrow[hook]{r}
		%\&
		%H \arrow[two heads]{r}
		%\&
		%K
		%\end{tikzcd}
	%\end{center}
	%\item A principal $H$-bundle $P$ over $L$
%\end{itemize}
%\end{block}
%\end{frame}


%\begin{frame}
%\begin{remark}[Classification of curved Yang-Mills gauge theories]
%If $\mathcal{G}$ acts faithfully on the normal bundle $\mathcal{T}$, preserving $L$, then a curved Yang-Mills gauge theory can be flattened if and only if there is flat lift .
%\end{remark}
%
%\begin{table}[h!]
		%\begin{tabularx}{\textwidth}{X X}
			%\rowcolor{gray}
			%Curved YM Gauge Theory & Singular Foliations $\mathcal{F}$ \\
			%Multiplicative Yang-Mills connection & $\mathcal{F}$-connection \\
			%\rowcolor{Gray}
			%Flat gauge theory & Flat singular foliation \\
			%Field redefinition of connection on $\mathcal{G}$ & Different choice of $\mathcal{F}$-connection
		%\end{tabularx}
%\end{table}
%\end{frame}

}

\begin{frame}[standout]
\thispagestyle{empty}
\topmargin -3.46 cm
\vspace*{\fill}
\begin{center}
\huge \textbf{Thank you!}
\end{center}
\vspace*{\fill}
\end{frame}

\appendix

\begin{frame}[fragile]{Classification of singular foliations}
\begin{block}{Theorem ([C.\ L.-G., S.-R.\ F.])}
\vspace{.5pt}
Formal singular foliations with leaf $L$ and transverse model $(\mathbb{R}^d, \tau_l)$ are equivalent to:
\begin{itemize}
	\item A Galois cover $L'$ over $L$ with structural group $K$
	\item A short exact sequence of groups
	\begin{center}
		\begin{tikzcd}[ampersand replacement=\&]
		\mathup{Inner}(\tau_l) \arrow[hook]{r}
		\&
		H \arrow[two heads]{r}
		\&
		K
		\end{tikzcd}
	\end{center}
	\item A principal $H$-bundle $P$ over $L$
\end{itemize}
\end{block}

\end{frame}

\end{document}

